% !TEX TS-program = xelatex
% !TEX encoding = UTF-8 Unicode
% !Mode:: "TeX:UTF-8"

\documentclass{resume}
\usepackage{zh_CN-Adobefonts_external} % Simplified Chinese Support using external fonts (./fonts/zh_CN-Adobe/)
%\usepackage{zh_CN-Adobefonts_internal} % Simplified Chinese Support using system fonts
\usepackage{linespacing_fix} % disable extra space before next section
\usepackage{cite}

\begin{document}
\pagenumbering{gobble} % suppress displaying page number

\name{梅王智汇}

%{E-mail}{mobilephone}{homepage}
% be careful of _ in emaill address
\contactInfo{(+86) 175-2009-0101}{maywzh@gmail.com}{软件研发工程师}{GitHub @maywzh}
% {E-mail}{mobilephone}
% keep the last empty braces!
%\contactInfo{xxx@yuanbin.me}{(+86) 131-221-87xxx}{}

% \section{\faGraduationCap\ 教育背景}
\section{教育背景}
\datedsubsection{\textbf{University of Wollongong},计算机科学与技术,\textit{联合培养硕士}}{2019.9 - 2021.4}
\datedsubsection{\textbf{华中师范大学},计算机科学与技术,\textit{硕士}}{2019.9 - 2021.6}
\ \textbf{(前20\%)},伍伦贡一等奖学金(2次),研究方向为基于教育知识图谱的推荐系统
\datedsubsection{\textbf{东南大学},信息工程,\textit{工学学士}}{2012.9 - 2016.6}
\ \textbf{(前20\%)},全国大学生机器人竞赛三等奖(2015)


% \section{\faCogs\ IT 技能}


% \end{itemize}

\section{全职/实习经历}

\datedsubsection{\textbf{多益网络有限公司}, 互联网开发工程师}{2016.7-2017.7}
\begin{itemize}
  \item \textbf{利用海量用户定位数据,对城市空间及人群移动特征进行研究。}第一个课题是基于香农熵和人群出行模式,构建城市网格与用户矩阵分析城市多样性/流动性分布;可视分析平台前端与可视化基于D3/Vue/Express开发,数据分析与存储采用Python/MySQL/MongoDB技术,为了均衡大数据情况下的页面可视化渲染消耗用canvas替代svg。第二个课题是对海量商场定位数据做人群分类与可视化查询,依据该系统撰写的论文被CIKM 2016(DAVA Workshop)录用,并收录于中科院软件所年会成果集
  \item 负责数据科学部HQ LAB的可视化原型开发,主导 TalkingMind 平台系统设计与前端开发
\end{itemize}
\datedsubsection{\textbf{北京君正有限公司}, 计算机视觉实习生}{2015.9-2015.10}\
\begin{itemize}
  \item \textbf{行人检测项目}第一个课题是基于OpenCV的行人检测项目,构建城市网格与用户矩阵分析城市多样性/流动性分布;可视分析平台前端与可视化基于D3/Vue/Express开发,数据分析与存储采用Python/MySQL/MongoDB技术,为了均衡大数据情况下的页面可视化渲染消耗用canvas替代svg。第二个课题是对海量商场定位数据做人群分类与可视化查询,依据该系统撰写的论文被CIKM 2016(DAVA Workshop)录用,并收录于中科院软件所年会成果集
  \item 负责数据科学部HQ LAB的可视化原型开发,主导 TalkingMind 平台系统设计与前端开发
\end{itemize}

% \begin{onehalfspacing}
% \end{onehalfspacing}

% \datedsubsection{\textbf{DID-ACTE} 荷兰莱顿}{2015年3月 - 2015年6月}
% \role{本科毕业设计}{LIACS 交换生}
% 利用结巴分词对中国古文进行分词与词性标注,用已有领域知识训练形成 classifier 并对结果进行调优
% \begin{onehalfspacing}
% \begin{itemize}
%   \item 利用结巴分词对中国古文进行分词与词性标注
%   \item 利用已有领域知识训练形成 classifier, 并用分词结果进行测试反馈
%   \item 尝试不同规则,对 classifier 进行调优
% \end{itemize}
% \end{onehalfspacing}

\section{竞赛获奖/项目作品}
% increase linespacing [parsep=0.5ex]
\begin{itemize}[parsep=0.2ex]
%   \item LeetCodeOJ Solutions, \textit{https://github.com/hijiangtao/LeetCodeOJ}
  \item 第三届中国软件杯大学生软件设计大赛\textbf{全国一等奖}( \textit{http://www.cnsoftbei.com/} ),2014 年8月
  \item 中国机器人大赛创意设计大赛\textbf{全国特等奖}( \textit{http://www.rcccaa.org/} ),2013年8月
%   \item 中国机器人大赛暨Robocup公开赛(武术擂台赛)全国一等奖,2013年10月
  \item 第11届北京理工大学“世纪杯”竞赛学生课外科技作品竞赛\textbf{特等奖},2013年8月
  \item VIS Components for security system, \textit{https://hijiangtao.github.io/ss-vis-component/}
  \item 个人博客:\textit{https://hijiangtao.github.io/},更多作品见 \textit{https://github.com/hijiangtao}
%   \item 电视节目"爸爸去哪儿"可视化分析展示, \textit{https://hijiangtao.github.io/variety-show-hot-spot-vis/}
\end{itemize}

% \section{\faHeartO\ 项目/作品摘要}
% \section{项目/作品摘要}
% \datedline{\textit{An Integrated Version of Security Monitor Vis System}, https://hijiangtao.github.io/ss-vis-component/ }{}
% \datedline{\textit{Dark-Tech}, https://github.com/hijiangtao/dark-tech/ }{}
% \datedline{\textit{融合社交网络数据挖掘的电视节目可视化分析系统}, https://hijiangtao.github.io/variety-show-hot-spot-vis/}{}
% \datedline{\textit{LeetCodeOJ Solutions}, https://github.com/hijiangtao/LeetCodeOJ}{}
% \datedline{\textit{Info-Vis}, https://github.com/ISCAS-VIS/infovis-ucas}{}
\section{技能栈}
% increase linespacing [parsep=0.5ex]
\begin{itemize}[parsep=0.2ex]
  \item \textbf{编程语言}: Go, Python, Java, SQL, C/C++, R, C\#, JavaScript, HTML/CSS
  \item \textbf{开发工具}: Linux, macOS, MySQL, MongoDB, Redis, Git, Latex, Docker
  \item \textbf{关键词}: 后端开发,Web开发,机器学习,知识图谱,DevOps
\end{itemize}

% \section{\faInfo\ 社会实践/其他}
\section{社区参与/实践其他}
% increase linespacing [parsep=0.5ex]
\begin{itemize}[parsep=0.2ex]
  \item 乐于参与开源社区讨论,\textbf{参与翻译多篇深度学习方向的论文}
  \item 作为青思麦田NLP方向的科研项目助教,与北大、华科的教授一同给本科生授课,包括网络科学,自然语言处理方向的课程
  \item 个人博客 https://maywzh.com 作品详见 Github: ht
\end{itemize}

%% Reference
%\newpage
%\bibliographystyle{IEEETran}
%\bibliography{mycite}
\end{document}
